% !Mode:: "TeX:UTF-8"

\documentclass[12pt,oneside]{book}

\newlength{\textpt}
\setlength{\textpt}{12pt}
    
\newcommand{\flypage}[1]{\begin{titlepage}\begin{center}\vspace*{\stretch{1}}#1\vspace*{\stretch{1}}\end{center}\end{titlepage}}
    
%========基本必备的宏包========%
\RequirePackage{calc,float,moresize}
%\RequirePackage[onehalfspacing]{setspace}
\linespread{1.5}
%1.3 onehalfspacing
%试卷或需要文字紧凑的
%1.6 doublespacing

%===========加入目录 某章或某节=====%
\makeatletter

\newcommand{\addchtoc}[1]{
        \cleardoublepage
        \phantomsection
        \addcontentsline{toc}{chapter}{#1}}

\newcommand{\addsectoc}[1]{
        \phantomsection
        \addcontentsline{toc}{section}{#1}}

%===========全文基本格式==========%
\setlength{\parskip}{1.6ex plus 0.2ex minus 0.2ex}   %段落間距
\setlength{\parindent}{\textpt * \real{2}}

%=========页面设置=========%
\RequirePackage[a4paper, %a4paper size 297:210 mm
  bindingoffset=10mm,%裝訂線
  top=35mm,  %上邊距 包括頁眉
  bottom=30mm,%下邊距 包括頁腳
  inner=10mm,  %左邊距or inner
  outer=10mm,  %右邊距or  outer
  headheight=10mm,%頁眉
  headsep=15mm,%
  footskip=15mm,%
  marginparsep=10pt, %旁註與正文間距
  marginparwidth=6em,includemp=true% 旁註寬度計入width%旁註寬度
  ]{geometry}

%color
\RequirePackage[table,svgnames]{xcolor}

%================字體================%
%设置数学字体
\RequirePackage{amssymb,amsmath}
\RequirePackage{stmaryrd}
\everymath{\displaystyle}

\RequirePackage{fontspec}
%設置英文字體
\setmainfont[Mapping=tex-text]{DejaVu Serif}
\setsansfont[Mapping=tex-text]{DejaVu Sans}
\setmonofont[Mapping=tex-text]{DejaVu Sans Mono}


%中文環境
\RequirePackage[]{xeCJK}
\xeCJKsetup{PunctStyle=plain}
\xeCJKDeclareSubCJKBlock{LIUSHISIGUA}{ "4DC0 -> "4DFF}
\setCJKmainfont[FallBack=DejaVu Serif, ItalicFont=TW-Kai,LIUSHISIGUA=DejaVu Sans]{Source Han Serif CN}
\setCJKsansfont[FallBack=DejaVu Sans]{Source Han Sans CN}
\setCJKmonofont[FallBack=DejaVu Sans Mono]{TW-Kai}


%%===============中文化=========%
\renewcommand\contentsname{目~录}
\renewcommand\listfigurename{插图目录}
\renewcommand\listtablename{表格目录}
\renewcommand\bibname{参~考~文~献}
\renewcommand\indexname{索~引}
\renewcommand\figurename{图}
\renewcommand\tablename{表}
\renewcommand\partname{部分}
\renewcommand\appendixname{附录}
\renewcommand{\today}{\number\year{}年\number\month{}月\number\day{}日}


%=======页眉页脚格式=========%
\RequirePackage{fancyhdr}   %頁眉頁腳
\RequirePackage{zhnumber}  %计数器中文化
\pagestyle{fancy}
\renewcommand{\sectionmark}[1]
{\markright{第\zhnumber{\arabic{section}}节~~#1}{}}

\fancypagestyle{plain}{%
    \fancyhf{}
    \renewcommand{\headrulewidth}{0pt}
    \renewcommand{\footrulewidth}{0pt}
    \fancyhf[HR]{\ttfamily \footnotesize \rightmark }
    \fancyhf[FR]{\thepage}}
\pagestyle{plain}


%=========章節標題設計=========%
\RequirePackage{titlesec}
%修改part
\titleformat{\part}{\huge\sffamily}{}{0em}{}
%修改chapter
\titleformat{\chapter}{\LARGE\sffamily}{}{0em}{}
%修改section
\titleformat{\section}{\Large\sffamily}{}{0em}{}
%修改subsection
\titleformat{\subsection}{\large\sffamily}{}{0em}{}
%修改subsubsection
\titleformat{\subsubsection}{\normalsize\sffamily}{}{0em}{}


%================目录===============%
%toc label to contents space   dynamic adjust
\RequirePackage{tocloft}%
\renewcommand{\numberline}[1]{%
  \@cftbsnum #1\@cftasnum~\@cftasnumb%
}

%==============超鏈接===============%
\RequirePackage[colorlinks=true,linkcolor=blue,citecolor=blue]{hyperref} %設置書簽和目錄鏈接等
\newcommand{\hlabel}[1]{\phantomsection \label{#1}}%某一小段的引用


%=================文字強調=========%
\RequirePackage{xeCJKfntef}

\let\oldemph\emph % Save emph in oldemph
\renewcommand{\emph}[1]{\textcolor{red}{#1}}  

%==================插入圖片=======%
\RequirePackage{wrapfig}
\RequirePackage{graphicx}
\graphicspath{{figures/}}
%change the caption style a little like 1-1
\renewcommand{\thefigure}{\arabic{chapter}-\arabic{figure}}


%==========其他宏包===========%
\RequirePackage{tikz} 
\usetikzlibrary{calc}

%========脚注=========%
\newcommand*\circled[1]{
\tikz[baseline=(char.base)]
\node[shape=circle,draw,inner sep=0.4pt,minimum size=4pt] (char) {#1};}

\newcommand*\circledarabic[1]{\circled{\arabic{#1}}}

\RequirePackage{perpage} %the perpage package
\MakePerPage{footnote} %the perpage package command

\renewcommand*{\thefootnote}{\protect\circledarabic{footnote}}

\renewcommand\@makefntext[1]{\vspace{5pt}\noindent
\makebox[20pt][c]{\fontsize{10pt}{12pt}\@thefnmark}
\fontsize{10pt}{12pt}\selectfont #1}

\setlength{\skip\footins}{20pt plus 10pt}
%main body 与脚注之间的距离


%framed环境
\RequirePackage{framed}


\RequirePackage{indentfirst} 

\makeatother



\title{增广贤文}
\author{增广贤文}
\hypersetup{
  pdfkeywords={},
  pdfsubject={制作者邮箱:a358003542@outlook.com},
  pdfcreator={万泽}}
  
\newcommand{\bookcover}[1]{\tikz[remember picture,overlay]{\node[inner sep=0] at (current page.center)
{\includegraphics[width=\paperwidth,height=\paperheight]{#1}}}} 
 

  
\begin{document}
\frontmatter 

\thispagestyle{empty}

\bookcover{book_cover.png}

\cleardoublepage

\flypage{感谢上天}


\addchtoc{编者言}
\chapter*{编者言}
增广贤文全名增广昔时贤文,是一部民间谚语集,最早的编著者不详,应成书于明代,后多有修订。其中最有名的清代儒生周希陶的修订版本。

从网上查阅的资料看来,增广贤文分为上下两集,其中下集大体是一样的,但上集有两个区别很明显的版本。编者分析最早版本的增广贤文应该只有上集,这里称之为古训增广贤文。然后另外一个版本的上集和下集合并起来的版本应该是周希陶所修订的版本,这里称之为新增广贤文。

最后几句为:
\begin{verse}
池塘积水须防旱,田地勤耕足养家。\\
根深不怕风摇动,树正无愁月影斜。
\end{verse}

当为古训增广贤文。

最后几句为:
\begin{verse}
贫寒休要怨,富贵不须骄。\\
善恶随人作,祸福自己招。
\end{verse}

当为新增广贤文的上篇部分。

将古训增广贤文和新增广贤文的下篇部分合并的做法是不可取的,编者这里主要对古训增广贤文进行了批注,新增广贤文放于附录部分供读者参考。

新增广贤文摘自维基文库 \href{https://zh.m.wikisource.org/zh-hans/%E5%A2%9E%E5%BB%A3%E8%B3%A2%E6%96%87}{新增广贤文} ,因编者精力有限,新增广贤文并没有仔细核对,这是放在这里权做参考,一切以维基文库上的更新和批注为准。%



\addchtoc{目录}
\setcounter{tocdepth}{2}    
\tableofcontents



\mainmatter
\part{古训增广贤文}
昔时贤文,诲汝谆谆,集韵增文\footnote{韵,美哉。文,文字也。见诸文字而得其美是也。},多见多闻。

观今宜鉴古,无古不成今。

知己知彼,将心比心。

酒逢知己饮,诗向会人\footnote{会作诗懂诗的人。}吟。

相识满天下,知心能几人。

相逢好似初相识,到老终无怨恨心。

近水知鱼性,近山识鸟音。

易涨易退山溪水,易反易覆小人心。

运去金成铁,时来铁似金,读书须用意,一字值千金\footnote{东西的贵贱随着时运而变动,价值千金的文字却很少变动,读书时须用心体会。}。

逢人且说三分话,未可全抛一片心。

有意栽花花不发,无心插柳柳成荫\footnote{原为阴,现在可以改为荫了。}。

画虎画皮难画骨,知人知面不知心。

钱财如粪土,仁义值千金\footnote{见多人心险恶,始知仁义千金,都是务实之言。}。

流水下滩非有意,白云出岫\footnote{(xiù)岫,山洞。}本无心\footnote{天下之事自然而成最美,顺势而为最宜。}。

当时若不登高望,谁信东流海洋深。

路遥知马力,事久见人心\footnote{更为人们熟知的是日久见人心,事久见人心更确切些。}。

两人一般心,无钱堪\footnote{(kān)堪,能。}买金,一人一般心,有钱难买针。

相见易得好,久住难为人。

马行无力皆因瘦,人不风流只为贫。

饶人不是痴汉\footnote{痴汉,笨蛋。},痴汉不会饶人。

是亲不是亲,非亲却是亲\footnote{第一个亲是亲戚的意思,第二个亲是亲人的意思。或者解为第一个亲是名分上的亲,第二个亲是实实在在的亲。}。

美不美,乡中水,亲不亲,故乡人。

莺花犹怕春光老,岂可教人枉度春。

相逢不饮空归去,洞口桃花也笑人。

红粉佳人休使老,风流浪子莫教贫。

在家不会迎宾客,出外方知少主人。

黄芩\footnote{各参考资料多为黄金,应为黄芩(qín),黄芩一种常见的中药材。}无假,阿魏\footnote{阿魏,一种名贵稀有的中药材。}无真\footnote{黄芩这种常见的中药材一般是真的,而像阿魏这样少见稀有的中药材则一般是假的。}。

客来主不顾,应恐是痴人。

贫居闹市无人问,富在深山有远亲。

谁人背后无人说,哪个人前不说人。

有钱道真语,无钱语不真。

不信但看筵中酒,杯杯先劝有钱人。

闹里有钱,静处安身。

来如风雨,去似微尘。

长江后浪推前浪,世上新人赶旧人。

近水楼台先得月,向阳花木早逢春。

莫道君行早,更有早行人。

莫信直中直,须防仁不仁\footnote{此句第一个直是看起来正直,第二个直是实在的正直。第一个仁是看起来仁义,第二个仁是实在的仁义。整句意思是不要相信真有正直之徒,须要提防表面仁义之辈。}。

山中有直树,世上无直人。

自恨枝无叶,莫怨太阳偏。

大家都是命,半点不由人。

一年之计在于春,一日之计在于寅, 一家之计在于和,一生之计在于勤。

责人之心责己,恕己之心恕人。

守口如瓶,防意如城\footnote{此处意指内心的意欲,私心杂欲。此句意思是防范内心的杂欲要如同守城一般时时戒备。}。

宁可人负我,切莫我负人\footnote{须防仁不仁,又有不做负人之事,国家又何尝不是如此,时刻操练军马而又遍施仁义之事。有人批判此为人性本恶论或者封建糟粕论之类,实在是虚伪之至。此等务实言论为金玉良言也。}。

再三须慎意\footnote{孔子曰:再思则可矣。},第一莫欺心\footnote{最重要的是莫欺瞒自己的本心。此句有阳明心学的韵味。}。

虎生犹可近,人熟不堪亲。

来说是非者,便是是非人。

远水难救近火,远亲不如近邻。

有茶有酒多兄弟,急难何曾见一人。

人情似纸张张薄,世事如棋局局新。

山中也有千年树,世上难逢百岁人。

力微休负重,言轻莫劝人。

无钱休入众,遭难莫寻亲\footnote{此处的亲又何单指亲戚,亲朋亲友也是。}。

平生莫作皱眉事,世上应无切齿人\footnote{皱眉事,指让人皱眉不悦不快之事。切齿,咬牙切齿。}。

士者国之宝,儒为席上珍。

若要断酒法,醒眼看醉人。

求人须求大丈夫,济人须济急时无。

渴时一滴如甘露,醉后添杯不如无。

久住令人贱,频来亲也疏。

酒中不语真君子,财上分明大丈夫。

出家如初,成佛有余\footnote{出家人能够保持那份修行的初心,成佛都是绰绰有余的。}。

\textbf{积金千两,不如明解经书。}

养子不教如养驴,养女不教如养猪。

有田不耕仓廪虚,有书不读子孙愚。仓廪虚兮岁月乏\footnote{乏,指物质财富短缺。},子孙愚兮礼义疏。

同君一席话,胜读十年书。

人不通今古,马牛如襟裾\footnote{(jīn jú),有成语襟裾马牛,穿着人衣的马牛,比喻那些没头脑无知之人。}。

茫茫四海人无数,哪个男儿是丈夫。

白酒酿成缘好客,黄金散尽为收书。

救人一命,胜造七级浮屠。

城门失火,殃及池鱼。

庭前生瑞草,好事不如无。

欲求生富贵,须下死工夫\footnote{拼死的工夫。}。

百年成之不足,一旦败之有余\footnote{天下事成之不易,毁之则很容易。}。

人心似铁,官法如炉\footnote{以前此句多暗指官府的屈打成招,现在推荐只取字面意思,即官法炼人心。}。善化不足,恶化有余\footnote{人心善化不易,恶化却很容易。}。

水太清则无鱼,人至察则无徒\footnote{徒,朋精粹而为徒——张衡·思玄赋,可知徒最好的翻译是朋党,比朋友关系更紧密一些。}。

知者减半,省者全无\footnote{天下的人啊有智慧的要减去一半,能够自我反省的基本上没有。}。

在家由父,出嫁从夫\footnote{女子在家要听父亲的,出嫁了要顺从丈夫。因旧唐书有在家从父,出嫁从夫,此处原文为出家,改为出嫁也是合适的。}。

痴人畏妇,贤女敬夫\footnote{傻瓜才怕老婆,好女子都是敬重丈夫的。}。

是非终日有,不听自然无。

宁可正而不足,不可邪而有余。

宁可信其有,不可信其无。

竹篱茅舍风光好,道院僧堂终不如。

命里有时终须有,命里无时莫强求。

道院迎仙客,书堂隐相儒\footnote{成仙的宾客,成相的儒士。}。庭栽栖凤竹,池养化龙鱼\footnote{栖息凤凰的竹子,能化为龙的鱼。这两句都是在谈论寻常地方也能出贵人。}。

结交须胜己,似我不如无。但看三五日,相见不如初\footnote{交朋友须要结交胜过自己的,和我差不多的还不如不结交。只和他相处很短的时间,就会发现还不如初次见面那会呢。}。

人情似水分高下,世事如云任卷舒。

会说说都是,不会说无理\footnote{原为无礼,改为无理是妥当的。}。

磨刀恨不利,刀利伤人指。求财恨不得,财多害自己\footnote{磨刀只恨不够锋利,太锋利了反而会误伤人的手指。求财只恨得不到,财太多反而会害了自己。}。

知足常足,终身不辱。知止常止,终身不耻\footnote{懂得知足之道的人常常自足,这样终身不受羞辱。懂得知止之道的人常常自止,这样终身不受耻辱。}。

有福伤财,无福伤己\footnote{钱财乃身外之物,有福气的人才伤财,没福气的人那就要伤害自己的身体了。}。

差之毫厘,失之千里\footnote{此是一个成语,形容某些事情一开始只有很小的差错,但后面会造成巨大的失误。此谚语告诫人们做事马虎不得。}。

若登高必自卑,若涉远必自迩\footnote{有成语:登高必自卑,行远必自迩(ěr)。登高必先从低处开始,行远必先从近处开始。此谚语告诫人们做事要脚踏实地。}。

三思而行,再思可矣。

使口不如自走\footnote{动嘴巴皮子还不如自己行动。},求人不如求己。

小时是兄弟,长大各乡里。

妒财莫妒食,怨生莫怨死\footnote{嫉妒别人钱财多可以但真没必要嫉妒别人饮食好。埋怨生者可以但真没必要还去埋怨死者。}。

人见白头嗔,我见白头喜。多少少年亡,不到白头死。

墙有缝\footnote{原文为逢,改为缝隙的缝是合适的。},壁有耳\footnote{此谚语的意思是你说的话做的事总有人知道的。}。

好事不出门,恶事传千里。

贼是小人,知过君子\footnote{贼是小人,但其智却胜过君子。}。

君子固穷,小人穷斯滥也\footnote{语出《论语·卫灵公》,君子能固守穷困,小人穷困就胡作非为起来了。斯,就。滥,胡作非为。}。

贫穷自在,富贵多忧。

不以我为德,反以我为仇。

宁向直中取,不可曲中求\footnote{此谚语典故出自《封神演义》中姜太公钓鱼一事,曰:吾宁在直中取,不向曲中求,不为锦鳞设,只钓王与侯。姜太公通过此番言语表明了自己要堂堂正正做事的决心。}。

人无远虑,必有近忧。

知我者谓我心忧,不知我者谓我何求\footnote{此谚语最早的出处是诗经的《黍(shǔ)离》,又有成语黍离之悲,盖黍离一诗表达的是亡国之悲。}。

晴天不肯去,只待雨淋头。

成事莫说,覆水难收\footnote{事情已成就不要再说了,就好比泼出去的水再难收回了。}。

是非只为多开口,烦恼皆因强出头。

忍得一时之气,免得百日之忧。

近来学得乌龟法,得缩头时且缩头。

惧法朝朝乐,欺公日日忧\footnote{畏惧法纪则天天快乐,欺瞒公堂则日日担忧。}。

人生一世,草生一春。黑发不知勤学早,看看又是白头翁。月到十五光明少,人到中年万事休\footnote{君不见增广还有一句,人老心未老,这里三句皆是劝人珍惜光阴之意,解为中年不能再成事的是误读了。}。

儿孙自有儿孙福,莫为儿孙作马牛。

人生不满百,常怀千岁忧。

今朝有酒今朝醉,明日愁来明日忧。

路逢险处难回避,事到头来不自由\footnote{事情临到头就由不得自己了。}。

药能医假病,酒不解真愁。

人贫不语,水平不流。

一家有女百家求,一马不行百马忧。

有花方酌酒,无月不登楼。

三杯通大道\footnote{大道理。},一醉解千愁。

深山毕竟藏猛虎,大海终须纳细流。

惜花须检点,爱月不梳头\footnote{此谚语意思众说纷纭,首先可以明确的是此谚语在借花月谈论男女恋爱之事,下面是我的解释。爱惜花朵赏花的时候就要检点自己的行为而不是对花朵东摸西抓,爱月之美赏月的时候就要专心赏月而不是一边赏月一边梳头东张西望。此谚语意在告诫男孩子追求女孩子的时候一要检点,二要专心。}。

大抵选他肌骨好,不擦红粉也风流。

受恩深处宜先退,得意浓时便可休。莫待是非来入耳,从前恩爱反为仇。

留得五湖明月在,不愁无处下金钩\footnote{事情的根本还在,害怕没有方法达到目的吗。}。

休别有鱼处,莫恋浅滩头\footnote{不要离开有鱼的地方,不要贪恋没鱼的浅水滩头。}。

去时终须去,再三留不住。

忍一句,息一怒,饶一着,退一步。

三十不豪,四十不富,五十将来寻死路。

生不论魂,死不认尸\footnote{活着的时候没必要讨论鬼魂,死了成了鬼魂也没必要去认你那尸体。}。

父母恩深终有别,夫妻义重也分离。

人生似鸟同林宿,大限\footnote{寿命大限。}来时各自飞。

人善被人欺,马善被人骑。

人无横财不富,马无野草不肥\footnote{增广的马一般指的是家马,横财对野草,新增广改为夜草是不对的。}。

人恶人怕天不怕,人善人欺天不欺\footnote{天道无亲,常与善人,善哉。}。善恶到头终有报,只争来早与来迟。

黄河尚有澄清日,岂可人无得运时。

得宠思辱,安居虑危。

念念有如临敌日,心心常似过桥时。

英雄行险道,富贵似花枝。

人情莫道春光好,只怕秋来有冷时。

送君千里,终须一别。

但将冷眼看螃蟹,看你横行到几时。

见事莫说,问事不知。

闲事休管,无事早归。

假缎染就真红色,也被旁人说是非。

善事可作,恶事莫为。

许人一物,千金不移。

龙生龙子,虎生豹儿。

龙游浅水遭虾戏,虎落平阳被犬欺。

一举首登龙虎榜,十年身到风凰池。

十年窗下无人问,一举成名天下知。

酒债寻常行处有,人生七十古来稀。

养儿待老,积谷防饥。

鸡豚狗彘之畜,无失其时。

数家之口,可以无饥矣。

常将有日思无日,莫把无时当有时。

时来风送腾王阁,运去雷轰荐福碑。

入门休问荣枯事,观看容颜便得知。

官清书吏瘦,神灵庙祝肥。

息却雷霆之怒,罢却虎狼之威。

饶人算人之本,输人算人之机。

好言难得,恶语易施。

一言既出,驷马难追。

道吾好者是吾贼,道吾恶者是吾师。

路逢侠客须呈剑,不是才人莫献诗。

三人同行,必有我师,择其善者而从之,其不善者而改之。

少壮不努力,老大徒悲伤。

人有善愿,天必佑之。

莫饮卯时酒,昏昏醉到酉。

莫骂酉时妻,一夜受孤凄。

种麻得麻,种豆得豆。

天眼恢恢,疏而不漏。

见官莫向前,做客莫在后。

宁添一斗,莫添一口。

螳螂捕蝉,岂知黄雀在后。

不求金玉重重贵,但愿儿孙个个贤。

一日夫妻,百世姻缘。

百世修来同船渡,千世修来共枕眠。

杀人一万,自损三千。

伤人一语,利如刀割。

枯木逢春犹再发,人无两度再少年。

未晚先投宿,鸡鸣早看天。

将相胸前堪走马,公候肚里好撑船。

富人思来年,穷人思眼前。

世上若要人情好,赊去物件莫取钱。

死生有命,富贵在天。

击石原有火,不击乃无烟。

为学始知道,不学亦徒然。

莫笑他人老,终须还到老。

但能依本分,终须无烦恼。

君子爱财,取之有道。

贞妇爱色,纳之以礼。

善有善报,恶有恶报。

不是不报,日子不到。

人而无信,不知其可也。

一人道好,千人传实。

凡事要好,须问三老。

若争小可,便失大道。

年年防饥,夜夜防盗。

学者如禾如稻,不学者如蒿如草。

遇饮酒时须饮酒,得高歌处且高歌。

因风吹火,用力不多。

不因渔父引,怎得见波涛。

无求到处人情好,不饮从他酒价高。

知事少时烦恼少,识人多处是非多。

入山不怕伤人虎,只怕人情两面刀。

强中更有强中手,恶人须用恶人磨。

会使不在家豪富,风流不用着衣多。

光阴似箭,日月如梭。

天时不如地利,地利不如人和。

黄金未为贵,安乐值钱多。

世上万般皆下品,思量唯有读书高。

世间好语书说尽,天下名山僧占多。

为善最乐,为恶难逃。

羊有跪乳之恩,鸦有反哺之义。

你急他未急,人闲心不闲。

隐恶扬善,执其两端。

妻贤夫祸少,子孝父心宽。

既坠釜甑,反顾无益。

翻覆之水,收之实难。

人生知足何时足,人老偷闲且是闲。

但有绿杨堪系马,处处有路透长安。

见者易,学者难。

莫将容易得,便作等闲看。

用心计较般般错,退步思量事事难。

道路各别,养家一般。

从俭入奢易,从奢入俭难。

知音说与知音听,不是知音莫与弹。

点石化为金,人心犹未足。

信了肚,卖了屋。

他人观花,不涉你目。

他人碌碌,不涉你足。

谁人不爱子孙贤,谁人不爱千钟粟。

莫把真心空计较,五行不是这题目。

与人不和,劝人养鹅。

与人不睦,劝人架屋。

但行好事,莫问前程。

河狭水急,人急计生。

明知山有虎,莫向虎山行。

路不行不到,事不为不成。

人不劝不善,钟不打不鸣。

无钱方断酒,临老始看经。

点塔七层,不如暗处一灯。

万事劝人休瞒昧,举头三尺有神明。

但存方寸土,留与子孙耕。

灭却心头火,剔起佛前灯。

惺惺常不足,懵懵作公卿。

众星朗朗,不如孤月独明。

兄弟相害,不如自生。

合理可作,小利莫争。

牡丹花好空入目,枣花虽小结实成。

欺老莫欺小,欺人心不明。

随分耕锄收地利,他时饱满谢苍天。

得忍且忍,得耐且耐。

不忍不耐,小事成大。

相论逞英雄,家计渐渐退。

贤妇令夫贵,恶妇令夫败。

一人有庆,兆民咸赖。

人老心未老,人穷志莫穷。

人无千日好,花无百日红。

杀人可恕,情理难容。

乍富不知新受用,乍贫难改旧家风。

座上客常满,樽中酒不空。

屋漏更遭连年雨,行船又遇打头风。

笋因落箨方成竹,鱼为奔波始化龙。

记得少年骑竹马,看看又是白头翁。

礼义生于富足,盗贼出于贫穷。

天上众星皆拱北,世间无水不朝东。

君子安平,达人知命。

忠言逆耳利于行,良药苦口利于病。

顺天者存,逆天者亡。

人为财死,鸟为食亡。

夫妻相合好,琴瑟与笙簧。

有儿贫不久,无子富不长。

善必寿老,恶必早亡。

爽口食多偏作药,快心事过恐生殃。

富贵定要安本分,贫穷不必枉思量。

画水无风空作浪,绣花虽好不闻香。

贪他一斗米,失却半年粮。

争他一脚豚,反失一肘羊。

龙归晚洞云犹湿,麝过春山草木香。

平生只会量人短,何不回头把自量。

见善如不及,见恶如探汤。

人贫志短,马瘦毛长。

自家心里急,他人未知忙。

贫无达士将金赠,病有高人说药方。

触来莫与说,事过心清凉。

秋至满山多秀色,春来无处不花香。

凡人不可貌相,海水不可斗量。

清清之水,为土所防。

济济之士,为酒所伤。

蒿草之下,或有兰香。

茅茨之屋,或有侯王。

无限朱门生饿殍,几多白屋出卿。

醉后乾坤大,壶中日月长。

万事皆已定,浮生空白茫。

千里送毫毛,礼轻仁义重。

一人传虚,百人传实。

世事明如镜,前程暗似漆。

光阴黄金难买,一世如驹过隙。

良田万倾,日食一升。

大厦千间,夜眠八尺。

千经万典,孝义为先。

一字入公门,九牛拖不出。

衙门八字开,有理无钱莫进来。

富从升合起,贫因不算来。

家中无才子,官从何处来。

万事不由人计较,一生都是命安排。

急行慢行,前程只有多少路。

人间私语,天闻若雷。

暗室亏心,神目如电。

一毫之恶,劝人莫作。

一毫之善,与人方便。

欺人是祸,饶人是福。

天眼恢恢,报应甚速。

圣贤言语,神钦鬼伏。

人各有心,心各有见。

口说不如身逢,耳闻不如目见。

养军千日,用在一朝。

国清才子贵,家富小儿骄。

利刀割体痕易合,恶语伤人恨不消。

公道世间唯白发,贵人头上不曾饶。

有钱堪出众,无衣懒出门。

为官须作相,及第必争先。

苗从地发,树向枝分。

父子和而家不退,兄弟和而家不分。

官有正条,民有和约。

闲时不烧香,急时抱佛脚。

幸生太平无事日,恐逢年老不多时。

国乱思良将,家贫思贤妻。

池塘积水须防旱,田地勤耕足养家。

根深不怕风摇动,树正无愁月影斜。

奉劝君子,各宜守己。

只此程式,万无一失。  
  
\part{附录}
\chapter{新增广贤文}
\section{上集}
昔时贤文,诲汝谆谆。集韵增广,多见多闻。

观今宜鑒古,无古不成今。

知己知彼,将心比心。

酒逢知己饮,诗向会人吟。相识满天下,知心能幾人?

相逢好似初相识,到老终无怨恨心。

近水知鱼性,近山识鸟音。

易涨易退山溪水,易反易覆小人心。

运去金成铁,时来铁似金。

读书须用意,一字値千金。

逢人且说三分话,未可全抛一片心。

有意栽花花不發,无心插柳柳成荫。

画虎画皮难画骨,知人知面不知心。

钱财如粪土,仁义値千金。

流水下滩非有意,白雲出岫本无心。

当时若不登髙望,谁信东流海洋深?

路遥知马力,日久见人心。

两人一般心,无钱堪买金;一人一般心,有钱难买针。

相见易得好,久住难为人。

马行无力皆因瘦,人不风流只为贫。

饶人不是痴汉,痴汉不会饶人。

是亲不是亲,非亲却是亲。

美不美,鄕中水;亲不亲,故鄕人。

莺花犹怕春光老,岂可教人枉度春?

相逢不饮空归去,洞口桃花也笑人。

红粉佳人休使老,风流浪子莫教贫。

在家不会迎宾客,出外方知少主人。

黄芩无假,阿魏无真。

客来主不顾,自是无良宾;良宾主不顾,应恐是痴人。

贫居闹市无人问,富在深山有远亲。

谁人背後无人说,哪个人前不说人?

有钱道真语,无钱语不真。

不信但看筵中酒,杯杯先敬有钱人。

鬧裏挣钱,静处安身。

来如风雨,去似微尘。

长江後浪推前浪,世上新人赶旧人。

近水楼臺先得月,向阳花木早逢春。

古人不见今时月,今月曾经照古人。

先到为君,後到为臣。

莫道君行早,更有早行人。

莫信直中直,须防仁不仁。

山中有直树,世上无直人。

自恨枝无葉,莫怨太阳偏。

万般皆是命,半点不由人。

一年之计在於春,一日之计在於晨;
一家之计在於和,一生之计在於勤。

责人之心责己,恕己之心恕人。

守口如甁,防意如城。

寧可人负我,切莫我负人。

再三须愼意,第一莫欺心。

虎生犹可近,人熟不堪亲。

来说是非者,便是是非人。

远水难救近火,远亲不如近邻。

有茶有酒多兄弟,急难何曾见一人?

人情似纸张张薄,世事如棋局局新。

山中也有千年树,世上难逢百歳人。

力微休负重,言轻莫劝人。

无钱休入众,遭难莫寻亲。

平生不做皱眉事,世上应无切齿人。

士乃国之宝,儒为席上珍。

若要断酒法,醒眼看醉人。

求人须求大丈夫,济人须济急时无。

渇时一滴如甘露,醉後添杯不如无。

久住令人贱,频来亲也疏。

酒中不语真君子,财上分明大丈夫。

贫贱之交不可忘,糟糠之妻不下堂。

出家如初,成佛有餘。

积金千两,不如明解经书。

养子不教如养驴,养女不教如养猪。

有田不耕仓廪虚,有书不读子孙愚。

仓廪虚兮歳月乏,子孙愚兮礼义疏。

聽君一席话,胜读十年书。

人不通今古,马牛如襟裾。

茫茫四海人无数,哪个男儿是丈夫?

白酒酿成縁好客,黄金散尽为收书。

救人一命,胜造七级浮屠。

城门失火,殃及池鱼。

庭前生瑞草,好事不如无。

欲求生富贵,须下死工夫。

百年成之不足,一旦败之有餘。

人心似铁,官法如炉。

善化不足,恶化有餘。

水太淸则无鱼,人至察则无徒。

知者减半,省者全无。

在家由父,出嫁从夫。

痴人畏妇,贤女敬夫。

是非终日有,不聽自然无。

寧可正而不足,不可邪而有餘。

寧可信其有,不可信其无。

竹篱茅舍风光好,道院僧堂终不如。

命裏有时终须有,命裏无时莫强求。

道院迎仙客,书堂隐相儒。

庭栽棲凤竹,池养化龙鱼。

结交须胜己,似我不如无。

但看三五日,相见不如初。

人情似水分髙下,世事如雲任巻舒。

会说说都是,不会说无礼。

磨刀恨不利,刀利伤人指。

求财恨不得,财多害自己。

知足常足,终身不辱。

知止常止,终身不耻。

有福伤财,无福伤己。

差之毫厘,失之千里。

若登髙必自卑,若渉远必自迩。

三思而行,再思可矣。

使口不如亲为,求人不如求己。

小时是兄弟,长大各鄕里。

妒财莫妒食,怨生莫怨死。

人见白头嗔,我见白头喜。

多少少年亡,不到白头死。

墙有缝,壁有耳。

好事不出门,恶事传千里。

若要人不知,除非己莫为。

为人不做亏心事,半夜敲门心不惊。

贼是小人,知过君子。

君子固穷,小人穷斯滥也。

贫穷自在,富贵多忧。

不以我为德,反以我为仇。

寧向直中取,不可曲中求。

人无远虑,必有近忧。

知我者谓我心忧,不知我者谓我何求?

晴天不肯去,只待雨淋头。

成事莫说,覆水难收。

是非只为多开口,烦恼皆因强出头。

忍得一时之气,免得百日之忧。

近来学得乌龟法,得缩头时且缩头。
惧法朝朝乐,欺公日日忧。


人生一世,草生一春。

黑髮不知勤学早,白首方悔读书迟。

月过十五光明少,人到中年万事休。

儿孙自有儿孙福,莫为儿孙作马牛。

为人莫做千年计,三十河东四十西。

人生不满百,常怀千歳忧。

今朝有酒今朝醉,明日愁来明日忧。

路逢险处难回避,事到头来不自由。

药能医假病,酒不解真愁。

人贫不语,水平不流。

一家有女百家求,一马不行百马忧。

有花方酌酒,无月不登楼。

三杯通大道,一醉解千愁。

深山毕竟藏猛虎,大海终须纳细流。

惜花须检点,爱月不梳头。

大抵选他肌骨好,不擦红粉也风流。

受恩深处宜先退,得意浓时便可休。

莫待是非来入耳,从前恩爱反为仇。

留得五湖明月在,不愁无处下金钩。

休别有鱼处,莫恋浅滩头。

去时终须去,再三留不住。

忍一句,息一怒,饶一著,退一歩。

三十不豪,四十不富,五十将来寻死路。

生不论魂,死不认屍。

一寸光阴一寸金,寸金难买寸光阴。

父母恩深终有别,夫妻义重也分离。

人生似鸟同林宿,大限来时各自飞。

人善被人欺,马善被人骑。

人无横财不富,马无野草不肥。

人恶人怕天不怕,人善人欺天不欺。

善恶到头终有报,只争来早与来迟。

黄河尚有澄淸日,岂可人无得运时。

得宠思辱,安居虑危。

唸-{念}有如临敌日,心心常似过桥时。

英雄行险道,富贵似花枝。

人情莫道春光好,只怕秋来有冷时。

送君千里,终须一别。

但将冷眼看螃蟹,看你横行到几时。

见事莫说,问事不知。

闲事休管,无事早归。

假缎染就真红色,也被旁人说是非。

善事可作,恶事莫为。

许人一物,千金不移。

龙生龙子,虎生豹儿。

龙游浅水遭虾戏,虎落平阳被犬欺。

一举首登龙虎榜,十年身到凤凰池。

十年窗下无人问,一举成名天下知。

酒债寻常行处有,人生七十古来稀。

养儿待老,积谷防饥。

鸡豚狗彘之畜,无失其时。数家之口,可以无饥矣。

当家才知盐米贵,养子方知父母恩。

常将有日思无日,莫把无时当有时。

树欲静而风不止,子欲养而亲不待。

时来风送滕王阁,运去雷轰荐福碑。

入门休问荣枯事,观看容颜便得知。

官淸书吏瘦,神灵庙祝肥。

息却雷霆之怒,罢却虎狼之威。

饶人算人之本,输人算人之机。

好言难得,恶语易施。

一言既出,驷马难追。

道吾好者是吾贼,道吾恶者是吾师。

路逢侠客须呈剑,不是才人莫献诗。

三人同行,必有我师焉,择其善者而从之,其不善者而改之。

欲昌和顺须为善,要振家声在读书。

少壮不努力,老大徒伤悲。

人有善愿,天必佑之。

莫饮卯时酒,昏昏醉到酉。

莫骂酉时妻,一夜受孤凄。

种麻得麻,种豆得豆。

天眼恢恢,疏而不漏。

见官莫向前,做客莫在后。

宁添一斗,莫添一口。

螳螂捕蝉,岂知黄雀在后。

不求金玉重重贵,但愿儿孙个个贤。

一日夫妻,百世姻縁。

百世修来同船渡,千世修来共枕眠。

杀人一万,自损三千。

伤人一语,利如刀割。

枯木逢春犹再发,人无两度再少年。

未晩先投宿,鸡鸣早看天。

将相胸前堪走马,公侯肚里好撑船。

富人思来年,穷人想眼前。

世上若要人情好,赊去物件莫取钱。

死生有命,富贵在天。

撃石原有火,不撃乃无烟。

为学始知道,不学亦徒然。

莫笑他人老,终须还到老。

和得邻里好,犹如拾片宝。

但能依本分,终须无烦恼。

大家做事寻常,小家做事慌张。

大家礼义教子弟,小家凶恶训儿郎。

君子爱财,取之有道。

贞妇爱色,纳之以礼。

善有善报,恶有恶报。

不是不报,日子不到。

万恶淫为首,百善孝当先。

人而无信,不知其可也。

一人道虚,千人传实。

凡事要好,须问三老。

若争小可,便失大道。

家中不和邻里欺,邻里不和说是非。

年年防饥,夜夜防盗。

学者是好,不学不好。

学者如禾如稻,不学者如蒿如草。

遇饮酒时须饮酒,得髙歌处且髙歌。

因风吹火,用力不多。

不因渔父引,怎得见波涛。

无求到处人情好,不饮从他酒价髙。

知事少时烦恼少,识人多处是非多。

入山不怕伤人虎,只怕人情两面刀。

强中更有强中手,恶人须用恶人磨。

会使不在家豪富,风流不用着衣多。

光阴似箭,日月如梭。

天时不如地利,地利不如人和。

黄金未为贵,安乐値钱多。

世上万般皆下品,思量唯有读书髙。

世间好语书说尽,天下名山僧占多。

为善最乐,作恶难逃。

羊有跪乳之恩,鸦有反哺之义。

孝顺还生孝顺子,忤逆还生忤逆儿。

不信但看檐前水,点点滴滴旧池窝。

你急他未急,人闲心不闲。

隐恶扬善,执其两端。

妻贤夫祸少,子孝父心宽。

既坠釜甑,反顾无益。

翻覆之水,收之实难。

人生知足何时足,人老偸闲且是闲。

但有绿杨堪系马,处处有路通长安。

见者易,学者难。莫将容易得,便作等闲看。

厌静还思喧,嫌喧又忆山。

自从心定后,无处不安然。

用心计较般般错,退歩思量事事难。

道路各别,养家一般。

从俭入奢易,从奢入俭难。

知音说与知音听,不是知音莫与弹。

点石化为金,人心犹未足。

信了赌,卖了屋。

他人观花,不渉你目。

他人碌碌,不渉你足。

谁人不爱子孙贤,谁人不爱千锺粟。

奈五行,不是这般题目。

莫把真心空计较,儿孙自有儿孙福。

书到用时方恨少,事非经过不知难。

天下无不是的父母,世上最难得者兄弟。

与人不和,劝人养鹅;与人不睦,劝人架屋。

但行好事,莫问前程。

河狭水激,人急计生。

明知山有虎,莫向虎山行。

路不行不到,事不为不成。

人不劝不善,钟不打不鸣。

无钱方断酒,临老始看经。

点塔七层,不如暗处一灯。

堂上二老是活佛,何用灵山朝世尊。

万事劝人休瞒昧,举头三尺有神明。

但存方寸土,留与子孙耕。

灭却心头火,剔起佛前灯。

惺惺常不足,懵懵作公卿。

众星朗朗,不如孤月独明。

兄弟相害,不如自生。

合理可作,小利莫争。

牡丹花好空入目,枣花虽小结实成。

欺老莫欺小,欺人心不明。

随分耕锄收地利,他时饱满谢苍天。

得忍且忍,得耐且耐。不忍不耐,小事成大。

相论逞英雄,家计渐渐退。

贤妇令夫贵,恶妇令夫败。

一人有庆,兆民咸赖。

人老心未老,人穷志莫穷。

人无千日好,花无百日红。

黄蜂一口针,橘子两边分。

世间痛恨事,最毒淫妇心。

杀人可恕,情理难容。

乍富不知新受用,乍贫难改旧家风。

座上客常满,樽中酒不空。

屋漏更遭连夜雨,行船又遇打头风。

笋因落箨方成竹,鱼为奔波始化龙。

记得少年骑竹马,看看又是白头翁。

礼义生于富足,盗贼出于贫穷。

天上众星皆拱北,世间无水不朝东。

士为知己者死,女为悦己者容。

色即是空,空即是色。

君子安贫,达人知命。

良药苦口利于病,忠言逆耳利于行。

顺天者存,逆天者亡。

有縁千里来相会,无縁对面不相逢。

有福者昌,无福者亡。

人为财死,鸟为食亡。

夫妻相合好,琴瑟与笙簧。

红粉易妆娇态女,无钱难作好儿郎。

有儿贫不久,无子富不长。

善必寿老,恶必早亡。

爽口食多偏作药,快心事过恐生殃。

富贵定要安本分,贫穷不必枉思量。

画水无风空作浪,绣花虽好不闻香。

贪他一斗米,失却半年粮。

争他一脚豚,反失一肘羊。

龙归晩洞云犹湿,麝过春山草木香。

平生只会量人短,何不回头把自量。

见善如不及,见恶如探汤。

人贫志短,马瘦毛长。

自家心里急,他人未知忙。

贫无达士将金赠,病有髙人说药方。

触来莫与说,事过心淸凉。

秋至满山多秀色,春来无处不花香。

凡人不可貌相,海水不可斗量。

淸淸之水,为土所妨。济济之士,为酒所伤。

蒿草之下,或有兰香。茅茨之屋,或有侯王。

无限朱门生饿殍,几多白屋出公卿。

醉里乾坤大,壶中日月长。

拂石坐来春衫冷,踏花归去马蹄香。

万事前身定,浮生空自忙。

千里送毫毛,礼轻仁义重。

叫月子规喉舌冷,宿花蝴蝶梦魂香。

一言不中,千言不用。

一人传虚,百人传实。

万金良药,不如无疾。

千里送鹅毛,礼轻情义重。

世事明如镜,前程暗似漆。

君子怀刑,小人怀惠。

架上碗儿轮流转,媳妇自有做婆时。

光阴黄金难买,一世如驹过隙。

良田万顷,日食一升。

大厦千间,夜眠八尺。

千经万典,孝义为先。

天上人间,方便第一。

一字入公门,九牛拔不出。

八字衙门向南开,有理无钱莫进来。

欲求天下事,须用世间财。

富从升合起,贫因不算来。

近河不得枉使水,近山不得枉烧柴。

家中无才子,官从何处来。

慈不掌兵,义不掌财。

一夫当关,万夫莫开。

万事不由人计较,一生都是命安排。

白云本是无心物,却被淸风引出来。

急行慢行,前程只有多少路。

命中只有如许财,丝毫不可有闪失。

人间私语,天闻若雷。

暗室亏心,神目如电。

一毫之恶,劝人莫作。一毫之善,与人方便。

欺人是祸,饶人是福。天眼恢恢,报应甚速。

圣贤言语,神钦鬼伏。

人各有心,心各有见。

口说不如身逢,耳闻不如目见。

见人富贵生欢喜,莫把心头似火烧。

养军千日,用在一朝。

国淸才子贵,家富小儿骄。

利刀割体痕易合,恶语伤人恨不消。

公道世间唯白发,贵人头上不曾饶。

有钱堪出众,无衣懒出门。

为官须作相,及第必争先。

苗从地发,树向枝分。

宅里燃火,烟气成云。

以直报怨,知恩报恩。

红颜今日虽欺我,白发他时不放君。

父子和而家不退,兄弟和而家不分。

一片云间不相识,三千里外却逢君。

官有正条,民有和约。

闲时不烧香,急时抱佛脚。

幸生太平无事日,恐逢年老不多时。

国乱思良将,家贫思贤妻。

池塘积水须防旱,田地勤耕足养家。

根深不怕风摇动,树正无愁月影斜。

争得猫儿,失却牛脚。

愚者千虑,必有一得,智者千虑,必有一失。

始吾于人也,听其言而信其行。

今吾于人也,听其言而观其行。

哪个梳头无乱发,情人眼里出西施。

珠沉渊而川媚,玉韫石而山辉。

夕阳无限好,只恐不多时。

久旱逢甘霖,他鄕遇故知;洞房花烛夜,金榜题名时。

惜花春起早,爱月夜眠迟。

掬水月在手,弄花香满衣。

桃红李白蔷薇紫,问著东君总不知。

教子教孙须教义,栽桑栽柘少栽花。

休念故鄕生处好,受恩深处便为家。

学在一人之下,用在万人之上。

一日为师,终生为父。

忘恩负义,禽兽之徒。

劝君莫将油炒菜,留与儿孙夜读书。

书中自有千锺粟,书中自有颜如玉。

莫怨天来莫怨人,五行八字命生成。

莫怨自己穷,穷要穷得干净;莫羡他人富,富要富得淸髙。

别人骑马我骑驴,仔细思量我不如,

待我回头看,还有挑脚汉。

路上有饥人,家中有剩饭。

积德与儿孙,要广行方便。

作善鬼神钦,作恶遭天遣。

积钱积谷不如积德,买田买地不如买书。

一日春工十日粮,十日春工半年粮。

疏懒人没吃,勤俭粮满仓。

人亲财不亲,财利要分淸。

十分伶俐使七分,常留三分与儿孙,

若要十分都使尽,远在儿孙近在身。

君子乐得做君子,小人枉自做小人。

好学者则庶民之子为公卿,不好学者则公卿之子为庶民。

惜钱莫教子,护短莫从师。

记得旧文章,便是新举子。

人在家中坐,祸从天上落。

但求心无愧,不怕有后灾。

只有和气去迎人,哪有相打得太平。

忠厚自有忠厚报,豪强一定受官刑。

人到公门正好修,留些阴德在后头。

为人何必争髙下,一旦无命万事休。

山髙不算髙,人心比天髙。

白水变酒卖,还嫌猪无糟。

贫寒休要怨,宝贵不须骄。

善恶随人作,祸福自己招。

奉劝君子,各宜守己。

只此呈示,万无一失。

\section{下集}
前人俗语,言浅理深。

补遗增广,集成书文。

世上无难事,只怕不专心。

成人不自在,自在不成人;

金凭火炼方知色,与人交财便知心。

乞丐无粮,懒惰而成。

勤俭为无价之宝,节粮乃众妙之门。

省事俭用,免得求人。

量大祸不在,机深祸亦深。

善为至宝深深用,心作良田世世耕。

群居防口,独坐防心。

体无病为富贵,身平安莫怨贫。

败家子弟挥金如土,贫家子弟积土成金。

富贵非关天地,祸福不是鬼神。

安分贫一时,本分终不贫。

不拜父母拜干亲,弟兄不和结外人。

人过留名,雁过留声。

择子莫择父,择亲莫择邻。

爱妻之心是主,爱子之心是亲。

事从根起,藕叶连心。

祸与福同门,利与害同城。

淸酒红人脸,财帛动人心!

宁可荤口念佛,不可素口骂人。

有钱能说话,无钱话不灵。

岂能尽如人意?但求不愧吾心。

不说自己井绳短,反说他人箍井深。

恩爱多生病,无钱便觉贫。

只学斟酒意,莫学下棋心。

孝莫假意,转眼便为人父母。

善休望报,回头只看汝儿孙!

口开神气散,舌出是非生!

弹琴费指甲,说话费精神。

千贯买田,万贯结邻。

人言未必犹尽,听话只听三分。

隔壁岂无耳,窗外岂无人?

财可养生须注意,事不关己不劳心。

酒不护贤,色不护病;

财不护亲,气不护命!

一日不可无常业,安闲便易起邪心!

炎凉世态,富贵更甚于贫贱;

嫉妒人心,骨肉更甚于外人!

瓜熟蒂落,水到渠成。

人情送匹马,买卖不饶针!

过头饭好吃,过头话难听!

事多累了自己,田多养了众人。

怕事忍事不生事自然无事;

平心静心不欺心何等放心!

天子至尊不过于理,在理良心天下通行。

好话不在多说,有理不在髙声!

一朝权在手,便把令来行。

甘草味甜人可食,巧言妄语不可听。

当场不论,过后枉然。

贫莫与富鬬,富莫与官争!

官淸难逃猾吏手,衙门少有念佛人!

家有千口,主事一人。

父子竭力山成玉,弟兄同心土变金。

当事者迷,旁观者淸。

怪人不知理,知理不怪人。

未富先富终不富,未贫先贫终不贫。

少当少取,少输当赢!

饱暖思淫欲,饥寒起盗心!

蚊虫遭扇打,只因嘴伤人!

欲多伤神,财多累心!

布衣得暖真为福,千金平安即是春。

家贫出孝子,国乱显忠臣!

宁做太平犬,莫做离乱人!

人有几等,官有几品。

理不卫亲,法不为民。

自重者然后人重,人轻者便是自轻。

自身不谨,扰乱四邻。

快意事过非快意,自古败名因败事。

伤身事莫做,伤心话莫说。

小人肥口,君子肥身。

地不生无名之辈,天不生无路之人。

一苗露水一苗草,一朝天子一朝臣。

读未见书如逢良友,见已读书如逢故人。

福满须防有祸,凶多料必无争。

不怕三十而死,只怕死后无名。

但知江湖者,都是薄命人。

不怕方中打死人,只知方中无好人。

说长说短,宁说人长莫说短;

施恩施怨,宁施人恩莫施怨。

育林养虎,虎大伤人。

冤家抱头死,事要解交人。

巻帘归乳燕,开扇出苍蝇。

爱鼠常留饭,怜蛾灯罩纱。

人命在天,物命在人。

奸不通父母,贼不通地邻。

盗贼多出赌博,人命常出奸情。

治国信谗必杀忠臣,治家信谗必疏其亲。

治国不用佞臣,治家不用佞妇。

好臣一国之宝,好妇一家之珍。

稳的不滚,滚的不稳。

儿不嫌母丑,狗不嫌家贫。

君子千钱不计较,小人一钱恼人心。

人前显贵,闹里夺争。

要知江湖深,一个不做声。

知止自当出妄想,安贫须是禁奢心。

初入行业,三年事成;

初吃馒头,三年口生。

家无生活计,坐吃如山崩。

家有良田万顷,不如薄艺在身;

艺多不养家,食多嚼不赢。

命中只有八合米,走遍天下不满升。

使心用心,反害自身。

国家无空地,世上无闲人。

妙药难医怨逆病,混财不富穷命人。

耽误一年春,十年补不淸;

人能处处能,草能处处生。

会打三班鼓,也要几个人。

人不走不亲,水不打不浑。

三贫三富不到老,十年兴败多少人!

买货买得真,折本折得轻;

不怕问到,只怕倒问。

人强不如货强,价髙不如口便。

会买买怕人,会卖卖怕人。

只只船上有梢公,天子足下有贫亲。

既知莫望,不知莫向。

在一行,练一行。

穷莫失志,富莫癫狂。

天欲令其灭亡,必先让其疯狂。

梢长人胆大,梢短人心慌。

隔行莫贪利,久炼必成钢。

甁花虽好艳,相看不耐长。

早起三光,迟起三慌。

未来休指望,过去莫思量;

时来遇好友,病去遇良方。

布得春风有夏雨,哈得秋风大家凉。

晴带雨伞,饱带饥粮。

满壶全不响,半壶响叮当。

久利之事莫为,众争之地莫往。

老医迷旧疾,朽药误良方;

该在水中死,不在岸上亡。

舍财不如少取,施药不如传方。

倒了城墙丑了县官,打了梅香丑了姑娘。

燕子不进愁门,耗子不钻空仓。

苍蝇不叮无缝蛋,谣言不找谨愼人。

一人舍死,万人难当。

人争一口气,佛争一炷香。

门为小人而设,锁乃君子之防。

舌咬只为揉,齿落皆因眶。

硬弩弦先断,钢刀刃自伤。

贼名难受,龟名难当。

好事他人未见讲,错处他偏说得长。

男子无志纯铁无钢,女子无志烂草无瓤。

生男欲得成龙犹恐成獐,生女欲得成凤犹恐成虎。

养男莫听狂言,养女莫叫离母。

男子失教必愚顽,女子失教定粗鲁。

生男莫教弓与弩,生女莫教歌与舞。

学成弓弩沙场灾,学成歌舞为人妾。

财交者密,财尽者疏。

婚姻论财,夫妻之道。

色娇者亲,色衰者疏。

少实胜虚,巧不如拙。

百战百胜不如无争,万言万中不如一默。

有钱不置怨逆产,冤家宜解不宜结。

近朱者赤,近墨者黑。

一个山头一只虎,恶龙难鬬地头蛇。

出门看天色,进门看脸色。

商贾买卖如施舍,买卖公平如积德。

天生一人,地生一穴。

家无三年之积不成其家,国无九年之积不成其国。

男子有德便是才,女子无才便是德。

有钱难买子孙贤,女儿不请上门客。

男大当婚女大当嫁,不婚不嫁惹出笑话。

谦虚美德,过谦即诈。

自己跌倒自己爬,望人扶持都是假。

人不知己过,牛不知力大。

一家饱暖千家怨,一物不见赖千家。

当面论人惹恨最大,是与不是随他说吧!

谁人做得千年主,转眼流传八百家。

满载芝麻都漏了,还在水里捞油花!

皇帝坐北京,以理统天下。

五百年前共一家,不同祖宗也同华!

学堂大如官厅,人情大过王法。

找钱犹如针挑土,用钱犹如水推沙!

害人之心不可有,防人之心不可无!

不愁无路,就怕不做。

须向根头寻活计,莫从体面下功夫!

祸从口出,病从口入。

药补不如肉补,肉补不如养补。

思虑之害甚于酒色,日日劳力上床呼疾。

人怕不是福,人欺不是辱。

能言不是真君子,善处方为大丈夫!

为人莫犯法,犯法身无主。

姊妹同肝胆,弟兄同骨肉。

慈母多误子,悍妇必欺夫!

君子千里同舟,小人隔墙易宿。

文钱逼死英雄汉,财不归身恰是无。

妻子如衣服,弟兄似手足。

衣服补易新,手足断难续。

盗贼怨失主,不孝怨父母。

一时劝人以口,百世劝人以书。

我不如人我无其福,人不如我我常知足!

捡金不忘失金人,三两黄铜四两福。

因祸得福,求赌必输。

一言而让他人之祸,一忿而折平生之福。

天有不测风云,人有旦夕祸福。

不淫当斋,淡饱当肉。

缓歩当车,无祸当福。

男无良友不知己之有过,女无明镜不知面之精粗。

事非亲做,不知难处。

十年易读举子,百年难淘江湖!

积钱不如积德,闲坐不如看书。

思量挑担苦,空手做是福。

时来易借银千两,运去难赊酒半壶。

天晴打过落雨铺,少时享过老来福。

与人方便自己方便,一家打墙两家好看。

当面留一线,过后好相见。

入门掠虎易,开口告人难。

手指要往内撇,家丑不可外传。

浪子出于祖无德,孝子出于前人贤。

货离鄕贵,人离鄕贱。

树挪死,人挪活。

在家千日好,出门处处难。

三员长者当官员,几个明人当知县?

明人自断,愚人官断。

人怕三见面,树怕一墨线。

邨夫硬似铁,光棍软如棉。

不是撑船手,怎敢拏篙竿!

天下礼仪无穷,一人知识有限。

一人不得二人计,宋江难结万人縁。

家有三亩田,不离衙门前,鄕间无强汉,衙门就饿饭。

人人依礼仪,天下不设官。

衙门钱,眼睛钱;

田禾钱,千万年。

诗书必读,不可做官。

为人莫当官,当官皆一般。

换了你我去,恐比他还贪。

官吏淸廉如修行,书差方便如行善。

靠山吃山,种田吃田。

吃尽美味还是盐,穿尽绫罗还是棉。

一夫不耕,全家饿饭,一女不织,全家受寒。

金银到手非容易,用时方知来时难。

先讲断,后不乱,免得藕断丝不断。

听人劝,得一半。

不怕慢,只怕站。

逢快莫赶,逢贱莫懒。

谋事在人,成事在天!

长路人挑担,短路人赚钱。

宁卖现二,莫卖赊三。

赚钱往前算,折本往后算。

小小生意赚大钱,七十二行出状元。

自己无运至,却怨世界难。

胆大不如胆小,心寛甚如屋寛。

妻贤何愁家不富,子孙何须受祖田。

是儿不死,是财不散。

财来生我易,我去生财难。

十月滩头坐,一日下九滩。

结交一人难上难,得罪一人一时间。

借债经商,卖田还债;

赊钱起屋,卖屋还钱。

修起庙来鬼都老,拾得秤来姜卖完。

不嫖莫转,不赌莫看。

节食以去病,少食以延年。

豆腐多了是包水,梢公多了打烂船。

无口过是,无眼过难。

无身过易,无心过难。

不会凫水怨河湾,不会犁田怨枷担。

他马莫骑,他弓莫挽。

要知心腹事,但听口中言。

宁在人前全不会,莫在人前会不全。

事非亲见,切莫乱谈。

打人莫打脸,骂人莫骂短。

好言一句三冬暖,话不投机六月寒。

人上十口难盘,帐上万元难还。

放债如施,收债如讨。

告状讨钱,海底摸盐。

衙门深似海,弊病大如天。

银钱莫欺骗,牛马不好变。

好汉莫被人识破,看破不値半文钱。

狗咬对头人,雷打三世冤。

不卖香烧无剩钱,井水不打不满边。

事寛则园,太久则偏。

髙人求低易,低人求髙难。

有钱就是男子汉,无钱就是汉子难。

人上一百,手艺齐全。

难者不会,会者不难。

生就木头造就船,砍的没得车的圆。

心不得满,事不得全。

鸟飞不尽,话说不完。

人无喜色休开店,事不遂心莫怨天。

选婿莫选田园,选女莫选嫁奁。

红颜女子多薄命,福人出在丑人边。

人将礼义为先,树将花果为园。

临危许行善,过后心又变。

天意违可以人回,命早定可以心挽。

强盗口内出赦书,君子口中无戏言。

贵人语少,贫子话多。

快里须斟酌,耽误莫迟春。

读过古华佗,不如见症多。

东屋未补西屋破,前帐未还后又拖。

今年又说明年富,待到明年差不多。

志不同己,不必强合。

莫道坐中安乐少,须知世上苦情多。

本少利微强如坐,屋檐水也滴得多。

勤俭持家富,谦恭受益多。

细处不断粗处断,黄梅不落靑梅落。

见钱起意便是贼,顺手牵羊乃为盗。

要做快活人,切莫寻烦恼。

要做长寿人,莫做短命事。

要做有后人,莫做无后事。

不经一事,不长一智。

宁可无钱使,不可无行止。

栽树要栽松柏,结交要结君子。

秀才不出门,能知天下事。

钱多不经用,儿多不耐死。

弟兄争财家不穷不止,妻妾争风夫不死不止。

男人有志,妇人有势。

夫人死百将临门,将军死一卒不至。

天旱误甲子,人穷误口齿。

百歳无多日,光阴能几时?

父母养其身,自己立其志。

待有馀而济人,终无济人之日;

待有闲而读书,终无读书之时。

此书传后世,句句必精读,其中礼和义,奉劝告世人。

勤奋读,苦发奋,走遍天涯如游刃。








% 编者:万泽
\end{document}


